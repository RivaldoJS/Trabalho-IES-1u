\documentclass[12pt,oneside,a4paper,article]{abntex2}
\usepackage[utf8]{inputenc} % Codificação do documento
\usepackage[T1]{fontenc}    % Seleção de código de fonte.
\usepackage[brazil]{babel}  % Idioma do documento
\usepackage{graphicx}       % Inclusão de gráficos
\usepackage{tabularx}       % Tabelas avançadas
\usepackage{amsmath}        % Melhorias em matemática
\usepackage{lipsum}         % Geração de texto dummy
\usepackage{authblk}

% Configurações específicas do abntex2
% Aqui você pode adicionar configurações específicas, como redefinições de comandos
% ou adições de novos pacotes que são essenciais para o seu documento.

% Carrega o pacote abntex2cite para citações
\usepackage[alf]{abntex2cite} % ou use [num] para citações numéricas

\usepackage[left=3cm,right=2cm,top=3cm,bottom=2cm]{geometry} % Margens
\usepackage{setspace}       % Espaçamento entre linhas
% %\usepackage{natbib}         % Formatação de bibliografia

% Informações de título
\title{\textbf{Seu Título Aqui}}
\author{Kauã Oliveira Seixas \thanks{kaua.seixas@ucsal.edu.br}}
\author{Rivaldo de Jesus Santos \thanks{rivaldo.santos@ucsal.edu.br}}
\author[1]{Vinícius Scola Santana \thanks{viniciusscola.santana@ucsal.edu.br}}
\author[1]{Eduardo Campos Aguiar \thanks{eduardo.aguiar@ucsal.edu.br}}
\author[1]{Cauã César Rodrigues Costa\thanks{cauacesar.costa@ucsal.edu.br} }
\author[1*]{Orientador: Elton Figueiredo da Silva \thanks{elton.figueiredo@pro.ucsal.br}}


\pagenumbering{gobble} % Remove numeração de página

\affil{
  Bacharelado em Engenharia de Software \par
  Escola de Tecnologias \par
Universidade Católica do Salvador (UCSAL) \par
Av. Prof. Pinto de Aguiar, 2589 Pituaçu, CEP: 41740-090 \par
Salvador/BA, Brasil
}

\affil[1]{\textit {\{kaua.seixas, rivaldo.santos, viniciusscola.santana
, eduardo.aguiar, cauacesar.costa\}@ucsal.edu.br}}
\affil[1*]{\textit {\{elton.figueiredo\}@pro.ucsal.edu.br}}




\date{Março 2025}



\ifthenelse{\equal{\ABNTEXisarticle}{true}}{%
\renewcommand{\maketitlehookb}{}
}{}

% Configurações de aparência do PDF final
\usepackage{hyperref} % para inserir links
 \hypersetup{
      colorlinks=false,       % false: boxed links; true: colored links
      pdfborder={0 0 0},      % remove as bordas ao redor dos links
 }

\renewcommand*{\Authsep}{, }
\renewcommand*{\Authand}{, }
\renewcommand*{\Authands}{, }
\renewcommand*{\Affilfont}{\normalsize\normalfont}
\renewcommand*{\Authfont}{\bfseries}    % make author names boldface    
\setlength{\affilsep}{2em}   % set the space between author and affiliation

\newsavebox\affbox


\begin{document}

\begin{center}
      \includegraphics[width=0.3\textwidth]{imagens-template/ucsal_logo.png}
\end{center}
{\let\newpage\relax\maketitle}

\clearpage
\pagenumbering{arabic} % Retoma a numeração normal
\begin{resumoumacoluna}
      \lipsum[1] % Gera um texto de exemplo

      \vspace{\onelineskip}

      \noindent
      \textbf{Palavras-chaves}: Transformação digital, software, inovação.
\end{resumoumacoluna}

\clearpage

\textual

% Apresentação do Tema, Contextualização 
\section{Introdução}
Muitas pessoas pensam que software é simplesmente outra palavra para programas de computador. No entanto, quando falamos de engenharia de software, não se trata apenas do programa em si, mas de toda a documentação associada e dados de configurações necessários para fazer esse programa operar corretamente.

A Documentação descreve cada parte do código-fonte, uma função, uma classe, um trecho ou módulo. Podemos dizer que a documentação consiste em um conjunto de manuais gerais e técnicos, podendo ser organizado em forma de textos e comentários, utilizando ferramentas do tipo dicionários, diagramas e
fluxogramas, gráficos, desenhos, dentre outros \cite{coelho2009documentaccao}. Desse modo, temos que, a documentação é de suma importância para o desenvolvimento de um software, servindo para diversos propósitos
como:

\begin{itemize}
      \item Facilitar o entendimento e a comunicação entre os envolvidos no projeto, como desenvolvedores,
            clientes, usuários, testadores, gerentes, etc;
      \item Auxiliar na definição do escopo, dos objetivos e das funcionalidades do software;
      \item Orientar o desenvolvimento, o teste e a implantação do software;
      \item Permitir a verificação e a validação da conformidade do software com os requisitos especificados;
      \item Fornece instruções e orientações para o uso adequado do software pelos usuários finais;
      \item Apoiar a manutenção e a evolução do software ao longo do tempo.
\end{itemize}

\section{Fundamentos da Documentação de Software}
Quando falamos sobre documentação de software, estamos nos referindo a qualquer material textual que times de engenharia, teste, produto e demais profissionais utilizam para realizar o seu trabalho. a documentação deve ser uma descrição precisa sobre um sistema de software. Quanto maior a precisão desses documentos, maior o status de autoridade que eles podem ter.

A documentação pode ser dividida em dois grandes grupos: a parte técnica é considerada mais simples, pois descreve o trabalho do desenvolvedor, enquanto
que a parte para o usuário é a mais exigente e requer habilidades especiais para a redação de manuais, inserção de screenshots, desenhos e outros elementos
gráficos \cite{coelho2009documentaccao}.

Dentro desses dois grupos existem diversos tipos de documentação, sendo cada tipo útil para situações diferentes, elas são:

\subsection{Documentação Técnica}
É aquela voltada para os profissionais que participam do projeto,como desenvolvedores, testadores, gerentes, etc. Ela inclui:

Documentação de requisitos: descreve as necessidades, as expectativas e as restrições dos clientes e dos usuários em relação ao software. Ela pode ser feita em diferentes níveis de detalhamento,
como visão geral, especificação funcional, especificação técnica, casos de uso, histórias de usuário.


Documentação de arquitetura/design: descreve a arquitetura, o design e a estrutura do software. Ela pode
incluir diagramas, modelos, padrões, componentes, interfaces, etc.


Documentação de código: descreve o funcionamento interno do código-fonte do software. Ela
pode incluir comentários, anotações, cabeçalhos, etc.


Documentação de teste: descreve os métodos, as ferramentas, os cenários, os casos e os resultados
dos testes realizados no software. Ela pode incluir planos de teste, roteiros de teste, relatórios de
teste, evidências de teste, etc.


Documentação de implantação: descreve os procedimentos e as configurações necessárias para
instalar e executar o software em um ambiente específico. Ela pode incluir guias de instalação

Documentação da API: é uma forma especializada de documentação
técnica que fornece detalhes sobre como interagir com a API do software. Inclui descrições de
métodos, parâmetros de entrada, formatos de saída e exemplo.

\subsection{Documentação de Usuário}
É um conjunto de de meteriais criados para ajudar os
usuários a entenderem, instalarem e configurarem um software. Ela é voltada a pessoas
que não necessariamente têm conhecimento técnico. ela inclui:
-Guia de início rápido: Passos básicos para a instalação e configuração de um software.
-Tutoriais Interativos: Passo a passo para tarefas específicas.

\subsection{Documentação Legal ou de Conformidade}
É aquela que descreve os aspectos jurídicos ou regulatórios como política de privacidade, termos de uso, licença de software(ex.: GPL, MIT, Apache).


\subsection{Documentação de Processos}
Está relacionada ao gerenciamento do desenvolvimento do software como: Plano de projeto(Cronogramas, recursos e fases do projeto), relat´rios de status, atas de reunião, etc.

\section{Boas práticas de documentação de software}
Por fim é necessário ter uma boa prática de documentação para a contrução de um software como:


\begin{itemize}
      \item Definir o público-alvo e o propósito da documentação. A documentação deve ser adaptada às
            necessidades e ao nível de conhecimento dos leitores, seja eles desenvolvedores, usuários finais,
            gerentes ou clientes. O propósito da documentação também deve ser claro, seja ele instruir,
            informar, persuadir ou avaliar.
      \item Escolher o formato e a ferramenta adequados para a documentação. A documentação pode ser
            apresentada em diferentes formatos, como texto, diagramas, vídeos, tutoriais ou exemplos de
            código. A escolha do formato depende do tipo de informação que se quer transmitir, da
            complexidade do assunto e da preferência do público.
      \item Seguir um padrão e uma estrutura para a documentação. A documentação deve seguir um padrão
            de estilo, linguagem e formatação que seja consistente, claro e objetivo. Isso facilita a leitura, a
            compreensão e a busca pela informação.
      \item Manter a documentação atualizada e revisada. A documentação deve refletir o estado atual do
            projeto, as mudanças realizadas e as decisões tomadas
      \item Obter feedback e melhorar a documentação. A documentação deve ser testada e avaliada pelos
            leitores para verificar se ela atende às suas expectativas, necessidades e dúvidas.

\end{itemize}
São, também, boas práticas para criar um documentação: Clareza e consistência, Abordagem centrada no público, Controle de versão e gerenciamento de mudanças, Colaboração entre equipes.
Assim a documentação de um software trás diversos benefícios como a melhor qualidade e confiabilidade de um software, a redução dos erros e falhas no funcionamento do software, o aumento de produtividade e a eficiência do desenvolvimento e entrega do software, além de aumentar a satisfação e a fidelização dos clientes e dos usuários. Por fim, cabe ao time de desenvolvimento decidir qual tipo de documentação faz mais sentido ser adotado. Essa decisão deve ser tomada de acordo com o contexto do projeto, as habilidades do time e o perfil de quem vai
consumir esse conteúdo.

\section{Responsabilidade Civil e Contratos}
Em um contexto legal, a documentação de software também pode ter implicações
relacionadas à responsabilidade civil e aos contratos. A documentação pode estabelecer claramente os termos e condições de uso de um software, indicando
responsabilidades e isenções de responsabilidade. Isso é especialmente importante quando um software é utilizado em ambientes comerciais ou sensíveis.
Exemplo Prático: Se um software é desenvolvido para uma empresa, o contrato de desenvolvimento pode exigir que a documentação seja fornecida de
maneira clara e abrangente. Isso ajuda a prevenir disputas legais no futuro,
como por exemplo, um cliente alegando que o software não está funcionando
conforme o prometido.

\section{Questões legais}

A documentação de software e as questões legais estão profundamente interligadas. A documentação não é apenas um recurso técnico para os desenvolvedores, mas também desempenha um papel crucial nas implicações legais que envolvem a criação, distribuição e uso de software. A documentação bem elaborada pode servir como um ponto de referência legal em diversos contextos,
ajudando a proteger direitos de propriedade intelectual, clarificar responsabilidades e fornecer garantias contratuais.

\subsection{LGPD (Lei Geral de Proteção de Dados - Brasil)}
Esta Lei dispõe sobre o tratamento de dados
pessoais, inclusive nos meios digitais, por pessoa natural ou por pessoa jurídica de direito público ou privado, com o objetivo de proteger os direitos fundamentais de liberdade e de privacidade e o livre desenvolvimento da personalidade da pessoa natural. Exigindo que documentações de software incluam políticas de privacidade e tratamento de dados \cite{LGPD}.

\section{Questões Legais Internacionais}

\subsection{GDPR (General Data Protection Regulation)}
É uma legislação da União Europeia que garante a proteção de dados pessoais e defende os direitos dos indivíduos, se aplica a dados de pessoas localizadas na União Europeia (residentes ou não), impondo regras a organizações que tratam de dados pessoas \cite{GDPR}. Ex.: Uma empresa do Brasil que atende e coleta dados de clientes da Alemanha deve seguir as regras do GDPR.

\subsection{ISO/IEC 26514}Ele descreve e define requisitos para documentação e como estabelecer quais informações os usuários precisam, como determinar o caminho que essas informações devem ser apresentadas e como preparar as informações e disponibilizá-las \cite{ISO26514}.

\section{Licenças de software}

A licença de uso é o instrumento com o qual o titular dos
direitos do software ou o distribuidor licenciado delimitam o uso, a aplicação e a sua utilidade para o usuário final \cite{nettoquestoes}. Existem diversos tipos de licenças de software, cada uma com suas próprias restrições e obrigações. Alguns exemplos de licenças de software livre e de código aberto incluem:

\subsection{GNU - General Public License (GPL)} Richard M. Stallman, criador da Free Software Foundation em 1984, lançou o projeto GNU que se trata de uma licença de software livre que dá a todos os usuários a liberdade de redistribuir e modificar o software GNU \cite{GPLv3}.

Em vez de colocar o software em domínio público, o Projeto GNU o torna [...] "copyleft"  que diz que qualquer um que distribui o software, com ou sem modificações, tem que passar adiante a liberdade de copiar e modificar novamente o programa. O copyleft garante que todos os usuários possuem liberdade \cite{nettoquestoes}.

\subsection{MIT License}
É uma licença de software livre que permite aos usuários usar, modificar e distribuir o software livremente, inclusive para fins comerciais além de oferecer proteção contra disputas de patentes. Ela exige apenas a inclusão de um aviso de direitos autorais \cite{MITLicense}.

\subsection{Apache License 2.0}
É uma licença de software de código aberto que permite aos usuários usar, modificar e distribuir software livremente, inclusive para fins comerciais. Ele exige menção clara na documentação sobre termos de uso modificações e redistribuição \cite{Apache2}.


\section{Desafios Legais na Documentação de Software}
1 - Conformidade com Leis Internacionais: Como mencionado, a conformidade com regulamentações como o GDPR é um dos maiores desafios atuais. A documentação de software precisa ser ajustada para garantir que todas as políticas de privacidade e de proteção de dados estejam adequadas à legislação vigente nos diversos países onde o software é utilizado.

2 - Licenciamento de Software: A documentação de software precisa ser rigorosa na especificação de licenças de uso. Com a proliferação de licenças open-source, como a Apache License ou GNU GPL, é importante que as implicações legais de cada tipo de licença sejam claras na documentação, para garantir que os desenvolvedores e usuários finais compreendam suas permissões e limitações.

3 - Propriedade de Código e Trabalhos Colaborativos: Em projetos colaborativos, especialmente aqueles envolvendo várias empresas ou desenvolvedores independentes, a documentação precisa abordar questões de propriedade intelectual, especificando quem detém os direitos sobre o código-fonte e os recursos criados. Contratos claros e documentação detalhada ajudam a evitar disputas sobre a propriedade do trabalho.

\section{Conclusão}
A documentação de software é uma parte vital do processo de desenvolvimento de software.
Ela garante que todas as partes interessadas tenham as informações necessárias para entender,
usar e manter o software de forma eficaz.


\newpage

% Formatação da bibliografia
%bibliographystyle{plain}
\bibliography{referencias} % Assume que você tem um arquivo referencias.bib

\end{document}