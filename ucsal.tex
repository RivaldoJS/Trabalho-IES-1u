\documentclass[12pt,oneside,a4paper,article]{abntex2}
\usepackage[utf8]{inputenc} % Codificação do documento
\usepackage[T1]{fontenc}    % Seleção de código de fonte.
\usepackage[brazil]{babel}  % Idioma do documento
\usepackage{graphicx}       % Inclusão de gráficos
\usepackage{tabularx}       % Tabelas avançadas
\usepackage{amsmath}        % Melhorias em matemática
\usepackage{lipsum}         % Geração de texto dummy
\usepackage{authblk}

% Configurações específicas do abntex2
% Aqui você pode adicionar configurações específicas, como redefinições de comandos
% ou adições de novos pacotes que são essenciais para o seu documento.

% Carrega o pacote abntex2cite para citações
\usepackage[alf]{abntex2cite} % ou use [num] para citações numéricas

\usepackage[left=3cm,right=2cm,top=3cm,bottom=2cm]{geometry} % Margens
\usepackage{setspace}       % Espaçamento entre linhas
% %\usepackage{natbib}         % Formatação de bibliografia

% Informações de título
\title{\textbf{Seu Título Aqui}}
\author{Kauã Oliveira Seixas \thanks{kaua.seixas@ucsal.edu.br}}
\author{Rivaldo de Jesus Santos \thanks{rivaldo.santos@ucsal.edu.br}}
\author[1]{Vinícius Scola Santana \thanks{viniciusscola.santana@ucsal.edu.br}}
\author[1]{Eduardo Campos Aguiar \thanks{eduardo.aguiar@ucsal.edu.br}}
\author[1]{Cauã César Rodrigues Costa\thanks{cauacesar.costa@ucsal.edu.br} }
\author[1*]{Orientador: Elton Figueiredo da Silva \thanks{elton.figueiredo@pro.ucsal.br}}


\pagenumbering{gobble} % Remove numeração de página

\affil{
  Bacharelado em Engenharia de Software \par
  Escola de Tecnologias \par
Universidade Católica do Salvador (UCSAL) \par
Av. Prof. Pinto de Aguiar, 2589 Pituaçu, CEP: 41740-090 \par
Salvador/BA, Brasil
}

\affil[1]{\textit {\{kaua.seixas, rivaldo.santos, viniciusscola.santana
, eduardo.aguiar, cauacesar.costa\}@ucsal.edu.br}}
\affil[1*]{\textit {\{elton.figueiredo\}@pro.ucsal.edu.br}}




\date{Março 2025}



\ifthenelse{\equal{\ABNTEXisarticle}{true}}{%
\renewcommand{\maketitlehookb}{}
}{}

% Configurações de aparência do PDF final
% \usepackage{hyperref} % para inserir links
 \hypersetup{
      colorlinks=false,       % false: boxed links; true: colored links
      pdfborder={0 0 0},      % remove as bordas ao redor dos links
 }

\renewcommand*{\Authsep}{, }
\renewcommand*{\Authand}{, }
\renewcommand*{\Authands}{, }
\renewcommand*{\Affilfont}{\normalsize\normalfont}
\renewcommand*{\Authfont}{\bfseries}    % make author names boldface    
\setlength{\affilsep}{2em}   % set the space between author and affiliation

\newsavebox\affbox


\begin{document}

\begin{center}
  \includegraphics[width=0.3\textwidth]{imagens-template/ucsal_logo.png}
\end{center}
{\let\newpage\relax\maketitle}

\clearpage
\pagenumbering{arabic} % Retoma a numeração normal
\begin{resumoumacoluna}
  \lipsum[1] % Gera um texto de exemplo

  \vspace{\onelineskip}

  \noindent
  \textbf{Palavras-chaves}: Transformação digital, software, inovação.
\end{resumoumacoluna}

\clearpage

\textual

% Apresentação do Tema, Contextualização 
\section{Introdução}
Muitas pessoas pensam que software é simplesmente outra palavra para programas de computador.
No entanto, quando falamos de engenharia de software, não se trata apenas do programa em si, mas de
toda a documentação associada e dados de configurações necessários para fazer esse programa operar
corretamente. A documentação nada mais é que um conjunto de informações que descreve as características, funcionamento, o uso e os requisitos de um software, dando apoio ao processo de construção dele. Desse modo, temos que, a documentação é
de suma importância para o desenvolvimento de um software, servindo para diversos propósitos
como:

\begin{itemize}
  \item Facilitar o entendimento e a comunicação entre os envolvidos no projeto, como desenvolvedores,
        clientes, usuários, testadores, gerentes, etc;
  \item Auxiliar na definição do escopo, dos objetivos e das funcionalidades do software;
  \item Orientar o desenvolvimento, o teste e a implantação do software;
  \item Permitir a verificação e a validação da conformidade do software com os requisitos especificados;
  \item Fornece instruções e orientações para o uso adequado do software pelos usuários finais;
  \item Apoiar a manutenção e a evolução do software ao longo do tempo.
\end{itemize}

\section{Fundamentos da Documentação de Software}
Quando falamos sobre documentação de software, estamos nos referindo a qualquer material textual
que times de engenharia, teste, produto
e demais profissionais utilizam para realizar o seu trabalho. a documentação deve ser uma descrição
precisa sobre um sistema de software.
Quanto maior a precisão desses documentos, maior o status de autoridade que eles podem ter.
Documentações devem abstrair a tecnicidade de um assunto, seja de implementações,
configurações, e focar no que é essencial para quem a utilizar, trazendo informações completas e
imprescindíveis para que se possa alcançar um objetivo e realizar uma ação.
Existem diversos tipos de documentação, sendo cada tipo útil para situações diferentes:


\subsection{Documentação Técnica}
É aquela voltada para os profissionais que participam do projeto,como desenvolvedores, testadores, gerentes, etc. Ela inclui:

Documentação de requisitos: descreve as necessidades, as expectativas e as restrições dos clientes e dos usuários em relação ao software. Ela pode ser feita em diferentes níveis de detalhamento,
como visão geral, especificação funcional, especificação técnica, casos de uso, histórias de usuário.


Documentação de projeto: descreve a arquitetura, o design e a estrutura do software. Ela pode
incluir diagramas, modelos, padrões, componentes, interfaces, etc.


Documentação de código: descreve o funcionamento interno do código-fonte do software. Ela
pode incluir comentários, anotações, cabeçalhos, etc.


Documentação de teste: descreve os métodos, as ferramentas, os cenários, os casos e os resultados
dos testes realizados no software. Ela pode incluir planos de teste, roteiros de teste, relatórios de
teste, evidências de teste, etc.


Documentação de implantação: descreve os procedimentos e as configurações necessárias para
instalar e executar o software em um ambiente específico. Ela pode incluir guias de instalação


\subsection{Documentação de usuário}
É um conjunto de de meteriais criados para ajudar os
usuários a entenderem, instalarem e configurarem um software. Ela é voltada a pessoas
que não necessariamente têm conhecimento técnico. ela inclui:
-Guia de início rápido: Passos básicos para a instalação e configuração de um software.
-Tutoriais Interativos: Passo a passo para tarefas específicas.

\subsection{Documentação de arquitetura/design}
A documentação de arquitetura ou design fornece um
modelo da estrutura do software, detalhando os componentes de alto nível, suas interações e os
padrões de design subjacentes. É crucial para a integração de novos desenvolvedores e para manter
a consistência em grandes projetos.

\subsection{Documentação da API}A documentação da API é uma forma especializada de documentação
técnica que fornece detalhes sobre como interagir com a API do software. Inclui descrições de
métodos, parâmetros de entrada, formatos de saída e exemplo.

\section{Boas práticas de documentação de software}
Por fim é necessário ter uma boa prática de documentação para a contrução de um software como:


\begin{itemize}
  \item Definir o público-alvo e o propósito da documentação. A documentação deve ser adaptada às
        necessidades e ao nível de conhecimento dos leitores, seja eles desenvolvedores, usuários finais,
        gerentes ou clientes. O propósito da documentação também deve ser claro, seja ele instruir,
        informar, persuadir ou avaliar.
  \item Escolher o formato e a ferramenta adequados para a documentação. A documentação pode ser
        apresentada em diferentes formatos, como texto, diagramas, vídeos, tutoriais ou exemplos de
        código. A escolha do formato depende do tipo de informação que se quer transmitir, da
        complexidade do assunto e da preferência do público.
  \item Seguir um padrão e uma estrutura para a documentação. A documentação deve seguir um padrão
        de estilo, linguagem e formatação que seja consistente, claro e objetivo. Isso facilita a leitura, a
        compreensão e a busca pela informação.
  \item Manter a documentação atualizada e revisada. A documentação deve refletir o estado atual do
        projeto, as mudanças realizadas e as decisões tomadas
  \item Obter feedback e melhorar a documentação. A documentação deve ser testada e avaliada pelos
        leitores para verificar se ela atende às suas expectativas, necessidades e dúvidas.

\end{itemize}
São, também, boas práticas para criar um documentação: Clareza e consistência, Abordagem centrada no público, Controle de versão e gerenciamento de mudanças, Colaboração entre equipes.
Assim a documentação de um software trás diversos benefícios como a melhor qualidade e confiabilidade de um software, a redução dos erros e falhas no funcionamento do software, o aumento de produtividade e a eficiência do desenvolvimento e entrega do software, além de aumentar a satisfação e a fidelização dos clientes e dos usuários. Por fim, cabe ao time de desenvolvimento decidir qual tipo de documentação faz mais sentido ser adotado. Essa decisão deve ser tomada de acordo com o contexto do projeto, as habilidades do time e o perfil de quem vai
consumir esse conteúdo

\section{Questões legais}

-LGPD(Lei Geral de Proteção de Dados - Brasil):Esta Lei dispõe sobre o tratamento de dados
pessoais, inclusive nos meios digitais, por pessoa natural ou por pessoa jurídica de direito público
ou privado, com o objetivo de proteger os direitos fundamentais de liberdade e de privacidade e o
livre desenvolvimento da personalidade
da pessoa natura. Exigindo que documentações de software incluam políticas de privacidade e
tratamento de dados.
2-GDPR(Regulamento Geral de Proteção de Dados):é crucial para o desenvolvimento de software,
pois garante a proteção de dados pessoais e defende os direitos dos indivíduos, impacta na
documentação de sistemas que coletam dados de cidadão europeus.
3-Apache License 2.0: é uma licença de software de código aberto permissiva que permite aos
usuários usar, modificar e distribuir software livremente, inclusive para fins comerciais. Ele exige
menção clara na documentação sobre termos de uso modificações e redistribuição.
4-Creative Commons recomenda: usa licenças de software livre e de código aberto para software,e
para licenciar documentação técnica.
5-ISO/IEC 26514:Ele descrevee define requisitos para documentação e como estabelecer quais
informações os usuários precisam, como determinar o caminho
que essas informações devem ser apresentadas e como preparar as informações e disponibilizá-las

\section{Desafios e soluções}
1-MANTER A DOCUMENTAÇÃO ATUALIZADA:Um dos maiores desafios é garantir que a
documentação reflita o estado atual do software.
Ferramentas automatizadas e auditorias regulares da documentação podem ajudar a manter as coisas
atuais.
2-INCENTIVAR A PARTICIPAÇÃO DOS DEVS:Os desenvolvedores frequentemente veem a
documentação como uma tarefa.
Incentivar a participação através de ferramentas colaborativas e integrar a documentação no
processo de desenvolvimento pode ajudar a aliviar esse problema.
3-GERENCIAR A "DÍVIDA" DA DOCUMENTAÇÃO:Assim como com o código, a
documentação pode acumular "dívida"
ao longo do tempo. Revisar e refatorar regularmente a documentação pode evitar que ela fique
desatualizada ou redundante.

\section{Conclusão}
A documentação de software é uma parte vital do processo de desenvolvimento de software.
Ela garante que todas as partes interessadas tenham as informações necessárias para entender,
usar e manter o software de forma eficaz.
\citeonline{boulic:91} teste.



% Formatação da bibliografia
%bibliographystyle{plain}
\bibliography{referencias} % Assume que você tem um arquivo referencias.bib

\end{document}